%% \item
% pdfplatex thesis_nakano
\documentclass[12pt]{optlab-bachelor}
\usepackage{amsfonts}
\usepackage{amsmath}
\usepackage[dvipdfmx]{graphicx}
% 各自変更するように
\def\年度{2018}
\def\氏名{中野 壱帥}
\def\学生番号{15715051}
\def\題目{多目的最短経路問題における
\\動的計画法に基づいた
\\拡張ベルマンフォード法の提案
}

\def\背題目{多目的最短経路問題における
\\動的計画法に基づいた拡張ベルマンフォード法の提案}
% 削除しないように!
\renewcommand{\bibname}{参考文献}

\begin{document}
\frontmatter % 削除しないように!
\chapter{はじめに}
\section{研究背景}
現代には道路ネットワークや通信ネットワークなど様々なネットワークが存在する.
これらのネットワークには無数の経路や組み合わせが存在するため最適化した解を求めたい.
しかし,これらのネットワークに対する最適化を行う場合,複数の目的関数を考慮することが必要である.
例えば,道路ネットワークでは目的地までの時間とコストを最小化する必要がある.
このように複数の目的関数値を最大化(最小化)する解を求める問題は多目的最適化問題と呼ばれている.
多目的最適化問題の中でも,最短経路を求める問題は多目的最短経路問題と呼ばれている.

多目的最適化問題においては,それぞれの目的関数がトレードオフの関係にある場合が存在し,
全ての目的関数値が最大(最小)となる最適解が存在するとは限らない.
あらかじめ各目的関数に対する比重を決めて解を求めると1つの解が適切に求めることができる.
しかし,探索者の各目的関数に対する比重が決まっていない場合や複数の探索者がいる場合,
求まった1つの解がそれぞれの探索者に対して適切な解ではない場合がある.
そこで,最適解になり得るパレート最適解の集合を求める.
探索者が1つの解を求めるとき,パレート最適解の集合を求めることにより解を選択する意思決定を容易にできる.
また,それぞれの探索者が最適解となり得る解の集合から解を選択できるので,
それぞれの探索者が最適な解を求めることができる.
一般にパレート最適解は膨大な数存在するので効率的に列挙することが必要になる.

単一目的最短経路問題には負の要素を考慮した解法が提案されているが,
調査した限り従来の多目的最短経路問題には負の要素を考慮した研究がなされていない.
単一目的最短経路問題において負の要素を考慮した場合,負のサイクルが存在し解が求められない場合がある.
しかし,多目的最短経路問題において負の要素を考慮した場合,1つの目的関数において負のサイクルが存在しても
他の目的関数に対する解を求められる場合がある.
そこで,負の要素を含む多目的最短経路問題において負のサイクルが存在しない目的関数に対する
効率的な解法を考える.

\section{研究目的}
本研究では,多目的最短経路問題において一般的に膨大な数存在するとされるパレート解を効率的に列挙する.
また,従来研究では負の要素を考慮していないため負のサイクル存在時の解の定義がされていない.
本研究では,オフライン環境における多目的最短経路問題に対して以下を目的とする.

\begin{description}
  \item[目的1:]
  多目的最短経路問題に対する効率的解法を行う.
\end{description}

多目的最適化問題におけるパレート解列挙の複雑さを踏まえて,解法の提案を行う.
また,計算機を用いて解法の実験的評価を行う.

\begin{description}
  \item[目的2:]
  負の要素を考慮した解法を行う.
\end{description}

負のサイクルが存在する場合の解を定義し効率的な解法の提案を行う.
また,計算機を用いて解法の実験的評価を行う.

\section{章構成}

  本論文の章構成は以下である.
  \begin{itemize}
  \item 第2章では,多目的最適化問題と最短経路問題に対する定義を紹介し,多目的最短経路問題の解に対する定義を行う.
  \item 第3章では,多目的最短経路問題に対する従来研究を紹介する.
  \item 第4章では,提案解法の紹介と実装における工夫,本研究に対する成果を述べる.
  \item 第5章では,結論として本研究の成果と今後の課題について述べる.
\end{itemize}

\chapter{諸定義}
この章では,多目的最適化問題と最短経路問題に対する定義を紹介する.
多目的最適化問題における解となるパレート解の説明をする.単一目的最短経路問題に対する解法を紹介する.
また,多目的最短経路問題の解に対する定義を行う.本研究における定式化をする.(負の要素を考慮した場合を含む)

\section{多目的最適化問題}
最適化問題とは与えられたインスタンスに対して実行可能な最適解を求める問題である.
最適化問題のインスタンスは特定の集合上で定義された実数値関数または整数値関数
$f : A \rightarrow \mathbb{R}$で定義される.
最小化問題の場合,$minf(x)$となる$x$を求める.
すなわち,$x_0 \in A : \forall x\in A , f(x_0) \leq f(x) $となる$x_0$を求める.
最大化問題の場合,$maxf(x)$となる$x$を求める.

多目的最適化問題とは複数の目的関数に対する最適化問題である.
多目的最適化問題のインスタンスは最適化目的の数が$n$である場合,
$n$個の実数値関数または整数値関数$f : A \rightarrow \mathbb{R}$で定義される.
すなわち,$\vec{f} = (f_1 , \ldots , f_n)$と表される.
最小化問題の場合,$min\vec{f(x)}$となる$x$を求める.
多目的最適化問題の場合,それぞれの目的関数がトレードオフの関係にある場合が存在し,
全ての目的関数値が最大(最小)となる最適解が存在するとは限らない.
(例:$f_1(x_0) < f_2(x_0) \land f_1(x_1) > f_2(x_1)$)
一般的に,多目的最適化問題はパレート最適解の集合を求める.
パレート最適解は複数の目的関数をそのまま考慮された解なので,
求めたい選好解を見つけることや挙動変数の関係を知ることが可能になる.

パレート最適解とは取りうる値の範囲を全て考慮した上で支配されない解である.
解 $x,y$ が以下の条件を満たすとき,$x$ は $y$ を支配する.
\begin{itemize}
\item $\forall i \in \{1,\ldots,k\},f_i(x) \le f_i(y)$
\item $\exists i \in \{1,\ldots,k\},f_i(x) < f_i(y)$
\item $k$:最適化目的の数,$f$:目的関数
\end{itemize}

\section{最短経路問題}
最短経路問題とは重み付きグラフの与えられた2つのノード間を結ぶ経路の中で,
重みが最小の経路を求める最適化問題である.
\begin{description}
  \item[最短経路問題の種類]
\end{description}
\begin{itemize}
\item 2頂点対最短経路問題:特定の2つのノード間の最短経路問題.
\begin{itemize}
  \item[入力:]重み付きグラフ,始点$s$,終点$t$
  \item[出力:]$s$から$t$への最短経路
\end{itemize}
\item 単一始点最短経路問題:特定の1つのノードから他の全ノードとの間の最短経路問題.
\begin{itemize}
  \item[入力:]重み付きグラフ,始点$s$
  \item[出力:]$s$から全頂点への最短経路
\end{itemize}
\item 全点対最短経路問題:グラフ内のあらゆる2ノードの組み合わせについての最短経路問題.
\begin{itemize}
  \item[入力:]重み付きグラフ
  \item[出力:]全頂点対間の最短経路
\end{itemize}
\end{itemize}
\begin{description}
  \item[最短経路問題の主な解法]
\end{description}

\begin{itemize}
  \item 幅優先探索

  始点から近い順に探索する.
  重みがない(すべての重みが1である)最短経路問題に使われる.
  通った辺の本数に応じて重みが決まるため探索によって発見した頂点は最短経路が決定する.
  計算時間は $O(E)$である.
  以下に無向グラフにおける単一始点最短経路問題の解法を示す.

  \begin{quote}
    \begin{description}
      \item[入力:] グラフ$G=(V,E)$,始点$s \in V$
      \item[出力:] $s$から全ての頂点への経路
      \item[Step 1.] $s \rightarrow Q$
      \item[Step 2.] $Q = \{\emptyset\}$になるまで以下の操作を行う.
      \begin{description}
        \item[Step 2-1.] $Q$の先頭にあるノード$v$を取り出す.
        \item[Step 2-2.] $u = \{u \in V \mid v,u \in e \land e \in E\}$が未探索のとき,
        $u \rightarrow Q$.
      \end{description}

      \item[Step 3.] 経路を出力する.
    \end{description}
  \end{quote}
  2頂点対最短経路問題の場合,終点が見つかった時点で探索を終了し,
  始点から終点への経路を出力する.
  入力が有向グラフの場合:
  \begin{quote}
    \begin{description}
      \item[Step 2-2.] $u = \{u \in V \mid e = (v,u) \land e \in E\}$が未探索のとき,
      $u \rightarrow Q$.
    \end{description}
  \end{quote}
\end{itemize}

\begin{itemize}
  \item ダイクストラ法

  重みが最小のノードを対象に探索していく.
  全ての重みが非負数であるグラフにおいて使われる.
  全ての重みが非負数の場合,探索したノードの中で重みが最小のものは
  その後の探索で更新されることはないので重みが決定する.
  計算時間は $O(V^2)$である.
  以下に無向グラフにおける単一始点最短経路問題の解法を示す.

  \begin{quote}
    \begin{description}
      \item[入力:] グラフ$G=(V,E)$,始点$s \in V$,$E$の各辺の長さ
      \item[出力:] $s$から全ての頂点への経路
      \item[Step 1.] $s_w = 0$とし,$v \in V/\{s\}$に対して$v_w = \infty$とする.
      \item[Step 2.] $s \rightarrow Q$
      \item[Step 3.] $Q = \{\emptyset\}$になるまで以下の操作を行う.
      \begin{description}
        \item[Step 3-1.] $v = \{ v \in Q \mid v_w \leq v'_w \land v' \in Q \}$
        を取り出す.
        \item[Step 3-2.] $e = \{ e \in E \mid v \in e \}$について以下の操作を行う.

        \begin{description}
          \item[Step 3-2-1.] $u = \{ u \in V \mid u \in e\}$となる$u$に対して,
          $u_w > v_w + e_w$を満たすとき以下の操作を行う.

          \begin{description}
            \item[Step 3-2-1-1.] $u_w = v_w + e_w$.
            \item[Step 3-2-1-2.] $u \notin Q$のとき,$u \rightarrow Q$.
          \end{description}
        \end{description}
      \end{description}

      \item[Step 4.] 経路を出力する.
    \end{description}
  \end{quote}
\end{itemize}
2頂点対最短経路問題の場合,終点が更新対象となった時点で探索を終了し,
始点から終点への経路を出力する.
入力が有向グラフの場合:
\begin{quote}
\begin{description}
\item[Step 3-2.] $u = \{ u \in V \mid (v,u) = e \land e \in E \}$となる$u$に対して,
$u_w > v_w + e_w$を満たすとき以下の操作を行う.
\begin{description}
  \item[Step 3-2-1.] $u_w = v_w + e_w$.
  \item[Step 3-2-2.] $u \notin Q$のとき,$u \rightarrow Q$.
\end{description}
\end{description}
\end{quote}

\begin{itemize}
  \item ベルマンフォード法

  頂点数を $|V|$ とした時,全辺を緩めることを単に $|V|-1$ 回繰り返す.
  全ての重みが実数であるグラフにおいて使われる.
  負の閉路が存在する場合は,探索終了時に負の閉路の存在有無を返す.
  計算時間は $O(EV)$である.
  以下に無向グラフにおける単一始点最短経路問題の解法を示す.

  \begin{quote}
    \begin{description}
      \item[入力:] グラフ$G=(V,E)$,始点$s \in V$,$E$の各辺の長さ
      \item[出力:] $s$から全ての頂点への経路,負の閉路の存在有無
      \item[Step 1.] $s_w = 0$とし,$v \in V/\{s\}$に対して$v_w = \infty$とする.
      \item[Step 2.] $|V|-1$ 回以下の操作を行う.
      \begin{description}
        \item[Step 2-1.] $e = {e \in E \mid v,u \in e}$となる$e$に対して,
        $u_w > v_w + e_w$を満たすとき以下の操作を行う.
        \begin{description}
          \item[Step 2-1-1.] $u_w = v_w + e_w$.
        \end{description}
      \end{description}
      \item[Step 3.] Step 2-1を行いノードの重みが更新された場合,
      負の閉路の存在を報告する.
      \item[Step 4.] 経路を出力する.
    \end{description}
  \end{quote}
  2頂点対最短経路問題の場合,単一始点最短経路問題を解き,
  始点から終点への経路を出力する.
  入力が有向グラフの場合:
  \begin{quote}
    \begin{description}
      \item[Step 2-1.] $e = {e \in E \mid (v,u) = e}$となる$e$に対して,
      $u_w > v_w + e_w$を満たすとき以下の操作を行う.
      \begin{description}
        \item[Step 2-1-1.] $u_w = v_w + e_w$.
      \end{description}
    \end{description}
  \end{quote}
\end{itemize}

\begin{itemize}
  \item ワーシャルフロイド法

  重み付き有向グラフにおいて全点対最短経路を多項式時間で解くアルゴリズム.
  3つの頂点a, b, cを選んで、a→b→cという道がa→cという道より短ければa→cの距離を更新する
  という操作を全ての頂点の組み合わせで繰り返して最短距離を確定させていく.
  負の値にも対応でき,計算時間は $O(V^3)$である.
  以下に全点対最短経路問題の解法を示す.

  \begin{quote}
    \begin{description}
      \item[入力:] グラフ$G=(V,E)$,始点$s \in V$,$E$の各辺の長さ
      \item[出力:] 全頂点対の経路
      \item[Step 1.] $s_w = 0$とし,$v \in V/\{s\}$に対して$v_w = \infty$とする.
      \item[Step 2.] 各$1<k<|V|$に対して以下の操作を行う.
      \begin{description}
        \item[Step 2-1.] 各$1<i<|V|$に対して以下の操作を行う.
        \begin{description}
          \item[Step 2-1-1.] 各$1<j<|V|$に対して以下の操作を行う.
          \begin{description}
            \item[Step 2-1-1-1.] $d_{i,j} > d_{i,k} + d{k,j}$を満たすとき,
            $d_{i,j} = d_{i,k} + d{k,j}$
          \end{description}
        \end{description}
      \end{description}

      \item[Step 3.] 経路を出力する.
    \end{description}
  \end{quote}
  2頂点間の距離がマイナスの場合,負の閉路が存在する.
\end{itemize}

\section{多目的最短経路問題}


-------------------------------------------------------------------------------------

\chapter{従来研究}

\section{1}

\section{2}

\chapter{解法の提案と実験的評価}

\section{提案解法}

\section{実装における工夫}

\section{提案解法の評価と分析}

\chapter{結論}

\section{研究成果}

\section{今後の課題}

\chapter*{謝辞}
本研究を進めるにあたり,指導教員の宋 少秋教授には研究に対する助言や熱心な指導
をしていただきましたことを心から感謝いたします.またゼミや日常で多くの
知識や示唆をいただいた研究室の先輩,
同期の方々に感謝いたします.

\begin{flushright}
  2018年1月31日 \氏名
\end{flushright}
\endmatter % 削除しないように!
\end{document}
